 %\iffalse
\let\negmedspace\undefined
\let\negthickspace\undefined
\documentclass[journal,12pt,twocolumn]{IEEEtran}
\usepackage{xparse}
\usepackage{cite}
\usepackage{amsmath,amssymb,amsfonts,amsthm}
\usepackage{algorithmic}
\usepackage{graphicx}
\usepackage{textcomp}
\usepackage{xcolor}
\usepackage{txfonts}
\usepackage{listings}
\usepackage{enumitem}
\usepackage{mathtools}
\usepackage{gensymb}
\usepackage{comment}
\usepackage[breaklinks=true]{hyperref}
\usepackage{tkz-euclide}
\usepackage{listings}
\usepackage{gvv}
\def\inputGnumericTable{}
\usepackage[latin1]{inputenc}
\usepackage{color}
\usepackage{array}
\usepackage{longtable}
\usepackage{calc}
\usepackage{multirow}
\usepackage{hhline}
\usepackage{ifthen}
\usepackage{lscape}

\newtheorem{theorem}{Theorem}[section]
\newtheorem{problem}{Problem}
\newtheorem{proposition}{Proposition}[section]
\newtheorem{lemma}{Lemma}[section]
\newtheorem{corollary}[theorem]{Corollary}
\newtheorem{example}{Example}[section]
\newtheorem{definition}[problem]{Definition}
\newcommand{\BEQA}{\begin{eqnarray}}
\newcommand{\EEQA}{\end{eqnarray}}
\newcommand{\define}{\stackrel{\triangle}{=}}
\theoremstyle{remark}
\newtheorem{rem}{Remark}
\begin{document}

\bibliographystyle{IEEEtran}
\vspace{3cm}

\title{GATE-CS.51}
\author{EE23BTECH11046 - Poluri Hemanth$^{*}$}
\maketitle
\textbf{Question:}
Consider the differential equation \\$\frac{d^2y}{dx^2}+8\frac{dy}{dx}+16y=0$ and the boundary conditions $y(0)=1$ and $\frac{dy}{dx}(0)=0$. The solution to equation is:
\textbf{Solution:}\\
\begin{table}[h!]
    % Change address in github
        \input{table4.tex}
        \caption{Parameters}
        \label{tab:es.47}
\end{table}\\
We use Laplace transorm in order to find solution of a second order differential equation
\begin{align}
	\frac{d^2y}{dx^2}+8\frac{dy}{dx}+16y&\Large\xleftrightarrow{\mathcal{L}}s^2Y(s)-sy(0)-y'(0)+8sY(s)-8y(0)+16Y(s)\\
	Y(s)(s^2+8s+16)&=s+8\\
	\Rightarrow Y(s)&=\frac{s+8}{s^2+8s+16}\\
	&=\frac{1}{s+4}+\frac{4}{(s+4)^2}\label{1ae.1}
\end{align}
For inversion of $Y(s)$ in partial fractions-
\begin{align}
	&\frac{b}{(s+a)^n}\Large\xleftrightarrow{\mathcal{L}^{-1}}\frac{b}{(n-1)!}\cdot x^{n-1} e^{-ax}\cdot u(x)\label{invae.1}
\end{align}
Where b, a are real numbers, we invert $Y(s)$ to get $y(x)$:-\\
\begin{align}
        &Y(s)\Large\xleftrightarrow{\mathcal{L}^{-1}} y(x)
\end{align}
From \eqref{1ae.1},\eqref{invae.1}
\begin{align}
	y(x)&=\frac{1}{0!} e^{-4x}\cdot u(x)+\frac{4}{1!}x\cdot e^{-4x}\cdot u(x)\\
	&=(1+4x)e^{-4x}u(x)
\end{align}
\\
\\
\\
\\
\\
\\
\\
\begin{figure}
    \centering
    \includegraphics[width=1\linewidth]{figure.png}
        \caption{Plot of y(x)}
    \label{fig:enter-label}
\end{figure}


\end{document}





