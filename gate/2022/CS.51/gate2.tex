% \iffalse
\let\negmedspace\undefined
\let\negthickspace\undefined
\documentclass[journal,12pt,twocolumn]{IEEEtran}
\usepackage{xparse}
\usepackage{cite}
\usepackage{amsmath,amssymb,amsfonts,amsthm}
\usepackage{algorithmic}
\usepackage{graphicx}
\usepackage{textcomp}
\usepackage{xcolor}
\usepackage{txfonts}
\usepackage{listings}
\usepackage{enumitem}
\usepackage{mathtools}
\usepackage{gensymb}
\usepackage{comment}
\usepackage[breaklinks=true]{hyperref}
\usepackage{tkz-euclide} 
\usepackage{listings}
\usepackage{gvv}                                        
\def\inputGnumericTable{}                                 
\usepackage[latin1]{inputenc}                                
\usepackage{color}                                            
\usepackage{array}                                            
\usepackage{longtable}                                       
\usepackage{calc}                                             
\usepackage{multirow}                                         
\usepackage{hhline}                                           
\usepackage{ifthen}                                           
\usepackage{lscape}

\newtheorem{theorem}{Theorem}[section]
\newtheorem{problem}{Problem}
\newtheorem{proposition}{Proposition}[section]
\newtheorem{lemma}{Lemma}[section]
\newtheorem{corollary}[theorem]{Corollary}
\newtheorem{example}{Example}[section]
\newtheorem{definition}[problem]{Definition}
\newcommand{\BEQA}{\begin{eqnarray}}
\newcommand{\EEQA}{\end{eqnarray}}
\newcommand{\define}{\stackrel{\triangle}{=}}
\theoremstyle{remark}
\newtheorem{rem}{Remark}
\begin{document}

\bibliographystyle{IEEEtran}
\vspace{3cm}

\title{GATE-CS.51}
\author{EE23BTECH11046 - Poluri Hemanth$^{*}$}
\maketitle
\textbf{Question:}
Consider the following recurrence:
\begin{align}
	f(1)\;\;&=\;\;1;\label{g2-1}\\
	 f(2n)\;\;&=\;2f(n)-1,\text{  for $n\geq$1;}\label{g2-2}\\
	 f(2n+1)\;\;&=\;2f(n)+1,\text{  for $n\geq$1.}\label{g2-3}
\end{align}
Then, which of thefollowing is/are \textbf{TRUE?}\\
(A) $f(2^n-1)\;=\;2^n-1$\\
(B) $f(2^n)\;=\;1$\\
(C) $f(5\cdot2^n)\;=\;2^{n+1}+1$\\
(D) $f(2^n+1)\;=\;2^n+1$\\
\hfill{[GATE-CS.51 2022]}\\
\textbf{Solution:}\\
(A)
let $f(2^k-1)=2^k-1$ for any $k\geq1$,\\
From \eqref{g2-3}
\begin{align}
	f(2^{k+1}-1)&=f(2(2^k-1)+1)\label{1}\\
	&=2f(2^k-1)+1\\
        &=2(2^k-1)+1\\
        &=2^{k+1}-1
\end{align}
For $n=1$ in statement (A),\\
From \eqref{g2-1},\eqref{g2-2}
\begin{align}
	f(2-1)=2-1
\end{align}
Hence $f(2^n-1)=2^n-1$ for $n\geq1$\\
So statement A is TRUE\\
(B)\\
\\Let $f(2^n)=1$ for any n$\geq$0\\
From \eqref{g2-2}
\begin{align}
	f(2^{n+1})&=2f(2^n)-1\\
	f(2^{n+1})&=1
\end{align}
For $n=0$,in statement (B),\\
From \eqref{g2-1},\eqref{g2-2}
\begin{align}
	f(2)&=2f(1)-1\\
	&=1
\end{align}
Hence $f(2^n)=1$ for every $n\geq0$ value.\\
So statement B is TRUE.\\
\\(C)\\
Let,$f(5\cdot2^n)=2^{n+1}+1$ be true for any n value,
From \eqref{g2-2}
\begin{align}
	f(5\cdot2^{n+1})&=f(2(5\cdot2^n))\\
	&=2f(5\cdot2^n)-1\\
        &=2(2^{n+1}+1)-1\\
        &=2^{n+2}+1
\end{align}
For $n=0$,in statement (C)\\From \eqref{g2-3}
\begin{align}
	f(5)&=f(2\cdot2+1)\\
        &=2f(2)+1\\
        &=3
\end{align}
Hence $f(5.2^n)=2^{n+1}+1$ for $n\geq1$\\
So statement C is TRUE.\\
(D)\\
From \eqref{g2-3}
\begin{align}
	f(2^n+1)&=f(2\cdot2^{n-1}+1)\label{3}\\
	&=2f(2^{n-1})+1
\end{align}
From statement (B),\eqref{3}
\begin{align}
	f(2^n+1)&=3\text{  (for n$\geq$1})
\end{align}
Hence statement D is FALSE.









\end{document}
