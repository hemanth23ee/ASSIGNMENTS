% \iffalse
\let\negmedspace\undefined
\let\negthickspace\undefined
\documentclass[journal,12pt,twocolumn]{IEEEtran}
\usepackage{xparse}
\usepackage{cite}
\usepackage{amsmath,amssymb,amsfonts,amsthm}
\usepackage{algorithmic}
\usepackage{graphicx}
\usepackage{textcomp}
\usepackage{xcolor}
\usepackage{txfonts}
\usepackage{listings}
\usepackage{enumitem}
\usepackage{mathtools}
\usepackage{gensymb}
\usepackage{comment}
\usepackage[breaklinks=true]{hyperref}
\usepackage{tkz-euclide} 
\usepackage{listings}
\usepackage{gvv}
\def\inputGnumericTable{}                                 
\usepackage[latin1]{inputenc}                                
\usepackage{color}                                            
\usepackage{array}                                            
\usepackage{longtable}                                       
\usepackage{calc}                                             
\usepackage{multirow}                                         
\usepackage{hhline}                                           
\usepackage{ifthen}                                           
\usepackage{lscape}

\newtheorem{theorem}{Theorem}[section]
\newtheorem{problem}{Problem}
\newtheorem{proposition}{Proposition}[section]
\newtheorem{lemma}{Lemma}[section]
\newtheorem{corollary}[theorem]{Corollary}
\newtheorem{example}{Example}[section]
\newtheorem{definition}[problem]{Definition}
\newcommand{\BEQA}{\begin{eqnarray}}
\newcommand{\EEQA}{\end{eqnarray}}
\newcommand{\define}{\stackrel{\triangle}{=}}
\theoremstyle{remark}
\newtheorem{rem}{Remark}
\begin{document}

\bibliographystyle{IEEEtran}
\vspace{3cm}


\title{NCERT DISCRETE 11.9.2.15}
\author{EE23BTECH11046 - Poluri Hemanth$^{*}$}
\maketitle
\textbf{Question:}
If \( \frac{a^n +b^n}{a^{n-1} + b^{n-1}}\)is A.M between $a$ and $b$, then find value of $n$.
\break
\textbf{Solution:}
\begin{table}[h!]
        \input{table.tex}
        \caption{parameters}
\end{table}
\\$x(n)=a+nd$    Where,
\begin{align}
	d&=\frac{b-a}{k+1},\\
	 &=\frac{b-a}{2}\label{0}\\
	x(1)&=\frac{x(0)^n +x(2)^n}{x(0)^{n-1} + x(2)^{n-1}}\label{r} 
\end{align}
Using $Z$ transform.
\begin{align}
	x(n)*u(n)&\Large\xleftrightarrow{\mathcal{Z}}X(Z)\\
	X(Z)&=\frac{a}{1-z^{-1}}+\frac{dz^{-1}}{(1-z^{-1})^2}\label{8}\\
\end{align}
From contour integration method
\begin{align}
	x(n)&=\frac{1}{2\pi j}\oint\,X(Z)z^{n-1}\,dz\\
	\Rightarrow x(1)&=\frac{1}{2\pi j}\oint\,X(Z)\,dz
\end{align}
According to Cauchy's Residue Theorm:\\
For a $y(n)$ such that, \\
\begin{align}
	y(n)&=\frac{1}{2\pi j}\oint\,Y(Z)\,dz\\
	&=\sum\limits_{k=1}^N\text{RES}(Y,a_k),\text{($N$ is no of poles of $Y(Z)$)},\label{13}\\ 
	\notag\text{Where,}\\
	\text{RES}(Y,a_k)&=\frac{1}{(m-1)!}\lim_{z\to a_k}\frac{d^{m-1}}{dz^{m-1}}[Y(Z)\cdot(z-a_k)^{m}]
\end{align}
From \eqref{8},\eqref{13}
\begin{align}	
	x(1)&=\lim_{z\to1}\frac{a}{1-z^{-1}}(z-1)+\lim_{z\to1}\frac{1}{1!}\frac{d}{dz}\left(\frac{d\cdot z^{-1}}{(1-z^{-1})^2}(z-1)^2\right)\\
	\Rightarrow x(1)&=a+d\\
	\notag\text{From \eqref{0}}\\
	 x(1)&=\frac{a+b}{2}
\end{align}
From \eqref{r}
\begin{align}
    \frac{x(0)^n +x(2)^n}{x(0)^{n-1} + x(2)^{n-1}}&= \frac{x(0)+x(2)}{2}  \\
    \Rightarrow x(0)^n+x(2)^n&=x(2)x(0)^{n-1}+x(0)x(2)^{n-1} \\
    \Rightarrow x(0)^{n-1}(x(0)-x(2))&=x(2)^{n-1}(x(0)-x(2))\label{2}
 \end{align}
\begin{align}
	\Rightarrow n
	\begin{cases}
		=1  &\text{if } a\neq b\\
		\in R &\text{if } a=b
	\end{cases}
\end{align}
\\
\\
\begin{figure}[h!]
	\centering
	\includegraphics[width=2.5\columnwidth]{final2.png}
	\caption{Plot of n in planes}
	\label{solution}
\end{figure}

 \end{document}
