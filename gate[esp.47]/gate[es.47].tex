 %\iffalse
\let\negmedspace\undefined
\let\negthickspace\undefined
\documentclass[journal,12pt,twocolumn]{IEEEtran}
\usepackage{xparse}
\usepackage{cite}
\usepackage{amsmath,amssymb,amsfonts,amsthm}
\usepackage{algorithmic}
\usepackage{graphicx}
\usepackage{textcomp}
\usepackage{xcolor}
\usepackage{txfonts}
\usepackage{listings}
\usepackage{enumitem}
\usepackage{mathtools}
\usepackage{gensymb}
\usepackage{comment}
\usepackage[breaklinks=true]{hyperref}
\usepackage{tkz-euclide} 
\usepackage{listings}
\usepackage{gvv}
\def\inputGnumericTable{}                                 
\usepackage[latin1]{inputenc}                                
\usepackage{color}                                            
\usepackage{array}                                            
\usepackage{longtable}                                       
\usepackage{calc}                                             
\usepackage{multirow}                                         
\usepackage{hhline}                                           
\usepackage{ifthen}                                           
\usepackage{lscape}

\newtheorem{theorem}{Theorem}[section]
\newtheorem{problem}{Problem}
\newtheorem{proposition}{Proposition}[section]
\newtheorem{lemma}{Lemma}[section]
\newtheorem{corollary}[theorem]{Corollary}
\newtheorem{example}{Example}[section]
\newtheorem{definition}[problem]{Definition}
\newcommand{\BEQA}{\begin{eqnarray}}
\newcommand{\EEQA}{\end{eqnarray}}
\newcommand{\define}{\stackrel{\triangle}{=}}
\theoremstyle{remark}
\newtheorem{rem}{Remark}
\begin{document}

\bibliographystyle{IEEEtran}
\vspace{3cm}

\title{GATE-ES.47}
\author{EE23BTECH11046 - Poluri Hemanth$^{*}$}
\maketitle
\textbf{Question:}Second order ordinary differential equation $\frac{d^2y}{dx^2}-\frac{dy}{dx}-2y=0$ has values 
$y=2$ and$\frac{dy}{dx}=1$ at $x=0$.The value of $y$ at $x=1$ is?($round\; off\;\: to\;\: three\;\: decimal\;\: places$)
\\
\textbf{Solution:}\\
We convert given second order differential equation to s domain using Laplace transform and solve for $Y(s)$ and take inversion to get $y(x)$.
\begin{table}[h!]
    % Change address in github
	\input{table2.tex}
        \caption{Parameters}
        \label{tab:es.47}
\end{table}


\begin{align}
    \frac{d^2y}{dx^2}-\frac{dy}{dx}-2y&\Large\xleftrightarrow{\mathcal{L}}s^2Y(s)-sy(0)-y'(0)-sY(s)+y(0)-2Y(s)\\
	Y(s)\left(s^2-s-2\right)&=2s+3\\
    \Rightarrow Y(s)&=\frac{2s+3}{s^2-s-2}\\
    \Rightarrow Y(s)&=\frac{7/3}{s-2}-\frac{1/3}{s+1}
\end{align}
For inversion of $Y(s)$ in partial fractions-
\begin{align}
    &\frac{b}{s+a}\Large\xleftrightarrow{\mathcal{L}^{-1}}be^{-ax}\label{inv}
\end{align}
Where b, a are real numbers, we invert $Y(s)$ to get $y(x)$:-\\
From \eqref{inv}
\begin{align}
    &Y(s)\Large\xleftrightarrow{\mathcal{L}^{-1}} y(x)
\end{align}
\begin{align}
	y(x)&=\frac{7}{3}e^{2x}-\frac{1}{3}e^{-x}\\
   \Rightarrow y(1)&=16.335
\end{align}\\
\\
\\
\\
\begin{figure}
    \centering
    \includegraphics[width=1\linewidth]{gate2.png}
	\caption{Plot of y(x)}
    \label{fig:enter-label}
\end{figure}






\end{document}
