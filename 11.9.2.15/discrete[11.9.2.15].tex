% \iffalse
\let\negmedspace\undefined
\let\negthickspace\undefined
\documentclass[journal,12pt,twocolumn]{IEEEtran}
\usepackage{xparse}
\usepackage{cite}
\usepackage{amsmath,amssymb,amsfonts,amsthm}
\usepackage{algorithmic}
\usepackage{graphicx}
\usepackage{textcomp}
\usepackage{xcolor}
\usepackage{txfonts}
\usepackage{listings}
\usepackage{enumitem}
\usepackage{mathtools}
\usepackage{gensymb}
\usepackage{comment}
\usepackage[breaklinks=true]{hyperref}
\usepackage{tkz-euclide} 
\usepackage{listings}
\usepackage{gvv}                                        
\def\inputGnumericTable{}                                 
\usepackage[latin1]{inputenc}                                
\usepackage{color}                                            
\usepackage{array}                                            
\usepackage{longtable}                                       
\usepackage{calc}                                             
\usepackage{multirow}                                         
\usepackage{hhline}                                           
\usepackage{ifthen}                                           
\usepackage{lscape}

\newtheorem{theorem}{Theorem}[section]
\newtheorem{problem}{Problem}
\newtheorem{proposition}{Proposition}[section]
\newtheorem{lemma}{Lemma}[section]
\newtheorem{corollary}[theorem]{Corollary}
\newtheorem{example}{Example}[section]
\newtheorem{definition}[problem]{Definition}
\newcommand{\BEQA}{\begin{eqnarray}}
\newcommand{\EEQA}{\end{eqnarray}}
\newcommand{\define}{\stackrel{\triangle}{=}}
\theoremstyle{remark}
\newtheorem{rem}{Remark}
\begin{document}

\bibliographystyle{IEEEtran}
\vspace{3cm}


\title{NCERT DISCRETE 11.9.2.15}
\author{EE23BTECH11046 - Poluri Hemanth$^{*}$}



\maketitle{\textbf{Question:}}
if \( \frac{a^n +b^n}{a^{n-1} + b^{n-1}}\)is A.M between $a$ and $b$, then find value of $n$.
\break

\maketitle{\textbf{Solution:}}
 As A.M between any two numbers $a$ and $b$ is average of those  numbers.\\
 let $a$,$b$ are terms in A.P $x(m)$,So\\
 $x(0)=a$ , $x(2)=b$ and $x(1)$=A.M.
 
 \begin{align}	 
      \frac{x(0)^n +x(2)^n}{x(0)^{n-1} + x(2)^{n-1}}= \frac{x(0)+x(2)}{2}  \\
      2(x(0)^n +x(2)^n) = x(0)^n +x(2)^n +x(2).x(0)^{n-1}+x(0).x(2)^{n-1} \\
      x(0)^n +x(2)^n = x(2).x(0)^{n-1}+x(0).x(2)^{n-1} \\
      x(0)^{n-1}.(x(0)-x(2))=x(2)^{n-1}(x(0)-x(2))
 \end{align}
 For $x(0) \neq x(2)$ \\
 $x(0)^{n-1}$=$x(2)^{n-1}$ \\
 $\Rightarrow$ n=1.\\
  For $x(0)=x(2)$\\
$\Rightarrow$ $n\in$ R  i.e $n$ is a real value.\newline
\begin{table}[h!]
    % Change address in github
    \input{tables/table.tex}
    \caption{Solution}
    \label{tab:11.9.2.15}
\end{table}

\end{document}
