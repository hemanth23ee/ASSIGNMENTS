% \iffalse
\let\negmedspace\undefined
\let\negthickspace\undefined
\documentclass[journal,12pt,twocolumn]{IEEEtran}
\usepackage{xparse}
\usepackage{cite}
\usepackage{amsmath,amssymb,amsfonts,amsthm}
\usepackage{algorithmic}
\usepackage{graphicx}
\usepackage{textcomp}
\usepackage{xcolor}
\usepackage{txfonts}
\usepackage{listings}
\usepackage{enumitem}
\usepackage{mathtools}
\usepackage{gensymb}
\usepackage{comment}
\usepackage[breaklinks=true]{hyperref}
\usepackage{tkz-euclide} 
\usepackage{listings}
\usepackage{gvv}
% this is place of gvv package                                        
\def\inputGnumericTable{}                                 
\usepackage[latin1]{inputenc}                                
\usepackage{color}                                            
\usepackage{array}                                            
\usepackage{longtable}                                       
\usepackage{calc}                                             
\usepackage{multirow}                                         
\usepackage{hhline}                                           
\usepackage{ifthen}                                           
\usepackage{lscape}

\newtheorem{theorem}{Theorem}[section]
\newtheorem{problem}{Problem}
\newtheorem{proposition}{Proposition}[section]
\newtheorem{lemma}{Lemma}[section]
\newtheorem{corollary}[theorem]{Corollary}
\newtheorem{example}{Example}[section]
\newtheorem{definition}[problem]{Definition}
\newcommand{\BEQA}{\begin{eqnarray}}
\newcommand{\EEQA}{\end{eqnarray}}
\newcommand{\define}{\stackrel{\triangle}{=}}
\theoremstyle{remark}
\newtheorem{rem}{Remark}
\begin{document}

\bibliographystyle{IEEEtran}
\vspace{3cm}


\title{NCERT DISCRETE 11.9.2.15}
\author{EE23BTECH11046 - Poluri Hemanth$^{*}$}



\maketitle{\textbf{Question:}}
if \( \frac{a^n +b^n}{a^{n-1} + b^{n-1}}\)is A.M between a and b, then find value of n.
\break

\maketitle{\textbf{Solution:}}
 As A.M between any two numbers a and b is average of those  numbers.
 
 \begin{align}
      \frac{a^n +b^n}{a^{n-1} + b^{n-1}}= \frac{a+b}{2}  \\
      2(a^n +b^n) = a^n +b^n +b.a^{n-1}+a.b^{n-1} \\
      a^n +b^n = b.a^{n-1}+a.b^{n-1} \\
      a^{n-1}.(a-b)=b^{n-1}(a-b)
 \end{align}
 For a $\neq$ b \\
 a$^{n-1}$=b$^{n-1}$ \\
 $\Rightarrow$ n=1.\\
  For a=b\\
$\Rightarrow$ n$\in$ R  i.e n is a real value.\newline
\begin{table}[h!]
    % Change address in github
    \input{table.tex}
    \caption{Solution}
    \label{tab:11.9.2.15}
\end{table}

\end{document}
